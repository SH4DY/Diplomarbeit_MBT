\newglossaryentry {MBT} {
	name={MBT},
	description={Modellbasiertes Testen oder Model-based Testing},
	first={Model Based Testing (MBT)}
}

\newglossaryentry {SUT} {
	name={SUT},
	description={System Under Test. Das zu testende Softwaresystem.},
	first={System Under Test (SUT)}
}

\newglossaryentry {RFC} {
	name={Request-For-Change},
	description={Request For Change. Eine Änderungsanforderung bezeichnet einen  Wunsch nach Modifikation der Eigenschaften eines (Software-)Produkts.}
}

\newglossaryentry {Story} {
	name={User Story},
	description={Im Kontext von Scrum: Eine informell formulierte Anforderung für ein Softwareprodukt.}
}

\newglossaryentry {Backlog} {
	name={Scrum-Backlog},
	description={Im Kontext von Scrum: Eine Sammlung von Anforderungen (User Storys) an ein Softwareprodukt.}
}

\newglossaryentry {Review} {
	name={Code Review},
	description={Systematisches manuelles Prüfen von Code.}
}

\newglossaryentry {COS} {
	name={COS},
	description={Anforderung an ein Softwareprodukt in einer definierten Struktur.},
	first={Conditions of Satisfaction (COS)}
}

\newglossaryentry {SOA} {
	name={SoA},
	description={Softwarearchitekturmuster um Dienste zu modularisieren.},
	first={Service-orientierte Architektur}
	plural={Service-orientierte Architekturen}
}

\newglossaryentry {Test-First} {
	name={Test-First},
	description={Eine Vorgehensweise in der Programmierung, bei der Komponententests vor der eigentlichen Implementierung geschrieben werden.}
}

\newglossaryentry {Unit-Test} {
	name={Unit-Test},
	description={Komponententest. Dynamisches Testen kleinster zusammenhängender Softwarebausteine.}
}

\newglossaryentry {FSM} {
	name={FSM},
	description={Finite-State Machine. Ein endlicher Automat ist ein Modell eines Verhaltens, indem nur Aktionen, Zustände und Zustandsübergange existieren.},
	first={Finite-State-Machine (FSM)}
}

\newglossaryentry {POC} {
	name={POC},
	description={Proof of Concept. Ein Machbarkeitsnachweis zeigt, die prinzipielle Durchführbarkeit eines Vorhabens.},
	first={Proof of Concept (POC)}
}

\newglossaryentry {BDD} {
	name={BDD},
	description={Behavior Driven Development. Verhaltensgesteuerte Entwicklung beschreibt eine Vorgehensweise bei der Spezifikation und Qualitätssicherung in der Softwareentwicklung bei der Anforderungen in einer bestimmten textuellen Form vorliegen und automatisiert werden.},
	first={Behavior Driven Development (BDD)}
}

\newglossaryentry {BDT} {
	name={BDT},
	description={Behavior Driven Testing oder Behavior Driven Test. Verhaltensgesteuertes Testen ist ein alternativer Begriff für BDD, wobei der Fokus auf den Testeigenschaften der Vorgehensweise liegen.},
	first={Behavior Driven Testing (BDT)}
}

\newglossaryentry {ISTQB} {
	name={ISTQB},
	description={International Software Testing Qualifications Board. Eine international anerkannte Zertifizierungsstelle für Softwaretester.},
	first={International Software Testing Qualifications Board (ISTQB)}
}

\newglossaryentry {Framework} {
	name={Framework},
	description={Ein Programmiergerüst, das an sich noch kein dezidiertes Softwareprodukt darstellt.}
}

\newglossaryentry {UML} {
	name={UML},
	description={Unified Modelling Language. Eine grafische Modellierungssprache zur Spezifikation, Konstruktion und Dokumentation von Software.},
	first={Unified Modelling Language}
}

\newglossaryentry {UTP} {
	name={UTP},
	description={UML Testing Profile. Eine Erweiterung von UML um Konzepte und Elemente für die Softwarequalitätssicherung.},
	first={UML Testing Profile (UTP)}
}

\newglossaryentry {datapool} {
	name={Data Pool},
	description={Data Pools sind das von UTP eingeführte Konzept für den Umgang mit Testdaten.},
	plural={Data Pools}
}

\newglossaryentry {datapartition} {
	name={Data Partition},
	description={UTP Data Partitions stellen Äquivalenzklassen innerhalb eines Data Pools dar.},
	plural={Data Partitions}
}

\newglossaryentry {Bug} {
	name={Bug},
	description={Ein Implementierungsfehler.}
}

\newglossaryentry {Front-End} {
	name={Front-End},
	description={Eine gängige Bezeichnung im Feld der Informationstechnologie um die Schichten eines Softwaresystems näher am Benutzer und der Eingabe zu bezeichnen.}
}

\newglossaryentry {Back-End} {
	name={Back-End},
	description={Eine gängige Bezeichnung im Feld der Informationstechnologie um die Schichten eines Softwaresystems näher am System und der Datenverarbeitung zu bezeichnen.}
}


\newglossaryentry {Smoke-Testing} {
	name={Smoke-Testing},
	description={Zeitlich kurzer und wenig vorbereiteter vertikaler (durch alle Schichten) Testdurchlauf um grundlegendste Funktionalität zu testen.}
}

\newglossaryentry {OMG} {
	name={OMG},
	description={Die Object Management Group ist ein Konsortium, das sich der Einführung von Standards im Bereich objektorientierte Programmierung gewidmet hat.}
}