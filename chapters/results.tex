%%%%%%%%%%%%%%%%%%%%%%%%%%%%%%%%%%%%%%%%%%%%%%%%%%%%%%%%%%%%%%%%%%%%%%%%
\chapter{Neuartige modellbasierte Teststrategie für große agile Projekte basierend auf Graphwalker}
\label{sec:results}
%%%%%%%%%%%%%%%%%%%%%%%%%%%%%%%%%%%%%%%%%%%%%%%%%%%%%%%%%%%%%%%%%%%%%%%%

Aufbauend auf den Erkenntnissen der Fallstudie (siehe Abschnitt \ref{sec:fallstudie}) wurde eine neuartige modellbasierte Teststrategie entwickelt um komplexe Softwaresysteme in agilen Projekten zu testen. Diese Strategie hat unter anderem folgende Eigenschaften und zielt auf die, in Abschnitt \ref{sec:schwachstellen_raiffeisen}, beschriebenen Schwachstellen ab:

\begin{enumerate}
\item Einfachere Einbindung der Fachbereiche und Minimierung der Spezifikations-/Implementationslücke.
\item Minderung des Wartungsaufwandes für die Testlogik
\item Eliminierung der Notwendigkeit eine hohe Anzahl von einzelnen Teställen zu definieren
\item Visualisierung der Spezifikation und Bewusstmachung des potenziellen Testaufwandes
\item Minimierung der Abhängigkeit zu einem Werkzeug durch Wahl eines quelloffenen Tools
\item Möglichkeit des punktuellen Einsatzes bzw. der inkrementellen Einführung
\item Erhöhung der Testabdeckung und damit Erhöhung der Qualität des Softwaresystems
\end{enumerate}

Im folgenden wird die genannte Teststrategie näher beschrieben.

\section{Aufbau der Teststufen}
Im Sinne von Test-First und einer hohen Priorisierung von Unit-Tests soll der Fokus des Testings auf drei Arten von Tests liegen:

\begin{itemize}
\item Eine breite Basis von Unit-Tests mit hoher Abdeckung
\item Modellbasierte Systemtests auf mehreren Ebenen mit direkter Verbindung zu Anforderungen/Changes
\item Manuelle System- und Abnahmetests
\end{itemize}

In Anlehung an Linz\cite{linz_testing_2014} (und mit dem Verweis auf die Diskussion in Abschnitt\ref{sec:discussion_unit}) soll die Unit-Test Ebene in Scrum-Projekten am breitesten sein\todo{Test PYRAMIDE einfügen}. Dies erscheint mit Hinblick auf die vielen kurzen Iterationen, die das Softwaresystem stark verändern, auch logisch. Unit-Tests lassen sich, weil sie am entwicklungsnächsten sind, auch am schnellsten modifizieren. Obwohl Unit-Tests einzelne Komponenten in Isolation testen, müssen sie zumindest eine verlässliche Aussagen treffen können, dass ebenjene Komponente ordnungsgemäß operiert. Diese Aussage wird möglich durch eine hohe Codeüberdeckung in Verbindung mit sorgfältig designten Testfällen. Auch auf Unit-Testebene gelten Best-Practices und Methodiken für hochqualitative Testfälle, wie sie Spillner und Linz beschreiben\cite{spillner_software_2014}. Die Besonderheit im Unit-Testing dieser Teststrategie liegt im Einsatz eines Behaviour-Driven-Testing\cite{chelimsky_rspec_2010} Frameworks und einer Testing-API. Für erstere Komponente existieren einige frei erhältliche Vertreter\footnote{Im Java-Umfeld bekannt sind die Frameworks \textit{jBehave} \url{http://jbehave.org/} und \textit{cucumber} \url{https://cucumber.io/}. In der Raiffeisen-Fallstudie wurde aber eine Eigenentwicklung als \textit{Proof of Concept} gemacht (siehe Abschnitt \ref{bdd} für eine kurze Erläuterung). Die erwähnte Testing-API \todo{Diagramm aus Damians Präsentation einfügen und markieren} bietet Schnittstellen zu Programmlogik und Umsystemen.\\
Im Sinne dieser Arbeit steht der modellbasierte Anteil der Teststrategie im Mittelpunkt. Es soll aber erwähnt sein, dass keine zwei Softwareprojekte gleich sind und sich die Gewichtungen der Testmethodiken im Detail unterscheiden \textit{sollten}! Der modellbasierte Anteil basiert auf 

\section{Details der Teststrategie und Architektur}

\subsection{Behaviour Driven Testing Framework}
\label{sec:bdd}

\subsection{Testing API}

\subsection{Modellbasierte Tests mit Graphwalker}