%GESTRICHEN
%%%%%%%%%%%%%%%%%%%%%%%%%%%%%%%%%%%%%%%%%%%%%%%%%%%%%%%%%%%%%%%%%%%%%%%%
\chapter{Zusammenfassung und Ausblick}
%\label{sec:conclusion}
%%%%%%%%%%%%%%%%%%%%%%%%%%%%%%%%%%%%%%%%%%%%%%%%%%%%%%%%%%%%%%%%%%%%%%%%
Die Komplexität moderner Softwaresysteme ist bereits sehr hoch und wird mit wachsender Rechenleistung und Konnektivität noch höher. Es braucht Strategien um der steigenden Komplexität in der Qualitätssicherung gerecht zu werden. Das Feld der modellbasierten Testtechnologien ist vielfältig. Diese Arbeit zeigt, dass \Gls{MBT} automatische Tests weiter stärken kann, nicht nur um wiederkehrende und sich wiederholende Tätigkeiten zu testen. Auch die vollständige Exploration der Programmzweige und die damit verbundene Gewissheit ermöglicht \Gls{MBT} so einfach wie keine andere Testmethodik.\\
Im Kapitel \ref{sec:problemdescription} \nameref{sec:problemdescription} wurden \Gls{MBT} Werkzeuge für den Einsatz auf Ebene des Komponententests und des Integrationstests erprobt. Das Werkzeug \textit{ModelJUnit} ist in der Verwendung an die weit verbreiteten Komponententestwerkzeuge angelehnt. Modelle werden als Java-Klasse dargestellt und Zustandsübergänge werden auf Methodenebene annotiert. Das Werkzeug bietet keine grafische Notation und erzeugt auch keine grafische Darstellung. Die Stärken von ModelJUnit liegen in der Einfachheit und in der guten Integrierbarkeit in bestehende Strukturen. Nachteilig wirkt sich aus, dass die Modelle weder die Verständlichkeit von grafischen Modellen, noch die ausdrucksstärke von textuellen Notationen besitzt.\\
Auf Integrationstestebene gab es Bemühungen den bekannten UML-Standard auszubauen und für die Definition von Testmodellen verwenden zu können. Das Ergebnis ist das UML Testing Profile. Dabei wurde die Sprache UML durch das Konzept der Profile erweitert und bietet nun Notationsmöglichkeiten speziell für das Testen. UTP bietet zwar einige essentiele Konzepte, die für den Gebrauch in komplexen Softwareprojekten nötig sind, aber wie die Generierung der Modelle erfolgen soll, bleibt den Werkzeugentwicklern vorbehalten. Eine Möglichkeit ist das manuelle Mapping von UTP Elementen zu JUnit, wie in Kapitel \ref{utp} beschrieben. Dies gestaltet sich aber als recht aufwendig und bietet wenig zusätzlichen Nutzen verglichen mit dem direkten Schreiben der JUnit Testfälle ohne den Umweg über UTP.\\
Auch das in Kapitel \ref{sec:graphwalker} beschriebene Werkzeug \textit{Graphwalker} lässt sich auf Integrationstestebene einsetzen. Es bietet einen Parser der das Einlesen von Modellen in Form von Graphen ermöglicht. Graphwalker generiert daraus Java-Code der mit Code für die Verbindung zum \Gls{SUT} gefüllt werden muss. Schlussendlich traviersiert Graphwalker das Modell auf die gewünschte Art und Weise. Das Werkzeug befähigt den Tester modellbasierte Tests gegen das \Gls{SUT} auszuführen ohne auf gewohnte Mittel zur Verbindungsherstellung mit dem \Gls{SUT} verzichten zu müssen. Schnittstellentests könnten beispielsweise weiterhin mit \textit{SoapUI}, Oberflächentests mit \textit{Selenium} realisiert werden.\\
Die Fallstudie \ref{sec:fallstudie} \nameref{sec:fallstudie} hat gezeigt, dass modellbasierte Tests sinnvoll in agilen Softwareprojekten eingesetzt werden können, auch ohne damit andere Testmethodiken zu verdrängen. Sogar eine schrittweise Einführung in ein großes bestehendes Softwareprojekt ist möglich, vorausgesetzt es werden flexible Werkzeuge genutzt. Der Einsatz von Graphwalker hat sich als sehr unkompliziert herausgestellt, weil wenige tiefgreifende Eingriffe in die bestehende Teststrategie gemacht werden müssen. Das Werkzeug kann unabhängig von der gegegebenen Umgebung genutzt werden. Es integriert sich, wenn gewünscht, auch in große Testmanagementprogramme von HP oder IBM. Dies ist durch die offene Java Schnittstelle möglich die Graphwalker mitbringt. Außerdem ist es nicht nötig bestehenden Adaptercode neu zu schreiben. Testfälle die als Java-Code vorliegen (z.B. Selenium) oder zumindest durch eine Java Schnittstelle ansprechbar sind (z.B. SoapUI) können in Graphwalker Tests eingebunden werden, weil Graphwalker selbst keine Annahme über das Ansprechen des \Gls{SUT} trifft.\\
%Schreiben wie die Kombination von BDD den Fachbereich einbindet und seine punktuellen Eigenschaften verliert
Für die weitere Festigung dieses Themengebiets wären mehrjährige Fallstudien großer und langdauernder Softwareprojekte höchst interessant. Durch den Vergleich von Qualitätsmetriken, aufgenommen während modellbasierte und nicht-modellbasierte Werkzeuge im Einsatz waren, ließen sich Schlüsse auf die tatsächliche Produktivitäts- und Qualitätssteigerung schließen die \Gls{MBT} verspricht.\\
Weiters bedarf es einer gewissen Standardisierung von Modellformaten und Werkzeugschnittstellen (siehe die Diskussion in Abschnitt \ref{sec:discussion_format}). Gegenwärtig werden Unternehmen eher abgeschreckt Ressourcen in eine modellbasierte Strategie zu investieren für die keine entsprechende Standardisierung der Sprache vorliegt. Mit dem Einstellen eines kritischen Werkzeugs könnten langjährige Bemühungen auf einen Schlag obsolet werden. Dieser Umstand zumindest durch müheloses Wiederverwenden von Testartefakten gemildert werden können. Das wiederum ist zur Zeit nicht möglich da die vielen erhältlichen Werkzeuge grundlegend verschiedene Ansätze verfolgen.







