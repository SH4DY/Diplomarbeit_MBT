%%%%%%%%%%%%%%%%%%%%%%%%%%%%%%%%%%%%%%%%%%%%%%%%%%%%%%%%%%%%%%%%%%%%%%%%
\chapter{Einleitung}
\label{sec:introduction}
%%%%%%%%%%%%%%%%%%%%%%%%%%%%%%%%%%%%%%%%%%%%%%%%%%%%%%%%%%%%%%%%%%%%%%%%

%=======================================================================
\section{Problemstellung}
%=======================================================================
\glsresetall
Steigende Komplexität in modernen technischen Disziplinen, darunter auch die Softwareentwicklung, erfordern Methoden zur Qualitätssicherung. Systematisches Prüfen ist ein zentrales Mittel, um Qualität von Software über einen längeren Zeitraum hinweg sicherzustellen \cite{spillner_software_2014}. Automatisierte Softwaretests auf Integrations- und Systemebene haben sich als effizienter Kontrollmechanismus etabliert \cite{dustin_software_2000}, sind aber herausfordernd in ihrer Umsetzung, vor allem in agilen Entwicklungsumgebungen. Kurze Iterationen, kombiniert mit einer hohen Frequenz von Änderungen in der Software erfordern spezialisierte Methoden \cite{linz_testing_2014}. Allein die Wahl des richtigen Werkzeugs für automatisiertes Testen garantiert noch keine erfolgreiche Teststrategie. Ungeachtet der gewählten Teststrategie, kann das Warten von Testfällen einen erheblichen Ressourcenaufwand verursachen der über Erfolg und Misserfolg eines Softwareprojekts entscheiden kann \cite{dustin_software_2000}. Vor allem agile Teams arbeiten auf multifunktionaler Basis. Deshalb darf sich auch die Generierung und Wartung von Testfällen nicht nur den technischen Rollen erschließen \cite{linz_testing_2014}. Bei mehreren weitverbreiteten Technologien ergibt sich aber genau diese Problematik.\\
%%%
%%%Ein vielversprechender Ansatz ist Testen auf Basis von Modellen, sogenannten %%%modellbasiertes Testen. Beim modellbasierten Testen wird die zu testende Applikation als %%% Modell in abstrahierter Form dargestellt. Daraus werden Testfälle automatisch generiert.
%%%%

%=======================================================================
\section{Motivation}
%=======================================================================
Kommerzielle Softwareprojekte sind teuer in der Entwicklung und der Wartung. Sie müssen unter hohem Zeitdruck geplant, spezifiziert, implementiert und ausgeliefert werden. Außerdem sollen solche Systeme dann oft mehrere Jahre betreut und in ihrer Funktionalität erweitert werden. Softwaresysteme die sensible Daten verarbeiten (Finanzwesen, Gesundheitswesen etc.) sind davon abhängig sehr exakte Ergebnisse zu liefern. Strenges Prüfen dieser ist unumgänglich. Gleichzeitig verknappen solche Tätigkeiten den engen zeitlichen Rahmen solcher Projekte. Deshalb wird, unabhängig von der Branche, das automatisierte regressive Testen von Software thematisiert \cite{graham_experiences_2012}.\\
Für jede gängige Programmiersprache und Plattform gibt es mächtige Test \Glspl{Framework}, die das Verhalten eines Benutzers automatisiert emulieren. In die Entwicklung von Testfällen mittels dieser \Glspl{Framework} fließt eine beträchtliche Anzahl an Ressourcen (laut \citeauthor{pol_management_2002} durchschnittlich 30\% - 40\% des Projektbudgets \cite{pol_management_2002}). Weiterhin ungelöst ist aber die Frage, wie es möglich ist diese Technologien intelligent, vielleicht sogar lernend (siehe Kapitel \ref{sec:test_based_modelling} \Newnameref{sec:test_based_modelling}), in die Entwicklung zu integrieren. Bis dahin scheint es aber noch ein langer Weg zu sein, wie es das \textit{LearnLib}-Projekt von \citeauthor{steffen_introduction_2011} zeigt \cite{steffen_introduction_2011}.\\

In der Zwischenzeit sind Ansätze nötig, die die Erkenntnisse des automatisierten Softwaretests übernehmen und neuartige Methoden zur Qualitätserhöhung und Wartungsreduzierung einbringen. Einer dieser Ansätze ist das modellbasierte Testen, welches eine Schicht zwischen den logischen und technischen Teil eines Tests zieht. Es findet also eine Abstrahierung auf Modelle des tatsächlichen Systems statt. Dies ermöglicht die wartungsintensiven Implementierungsdetails von den Abläufen des Tests zu entkoppeln, zu modularisieren und bestenfalls wieder zu gebrauchen.\\
Ein großflächiger Einsatz automatisierter Testtechnologien über längere Zeiträume hätte massive Auswirkungen auf die Landschaft der Softwarequalitätssicherung und damit der Softwareentwicklung:

\begin{itemize}
	\item Qualitätssicherungsbeauftragte könnten viel mehr Zeit darauf verwenden neue Funktionalität und Details zu testen. Interviews der Fallstudie in Kapitel \ref{sec:fallstudie} zeigten, dass durch die Automation der Hälfte der manuellen Tests ein Viertel der Wochenarbeitszeit eingespart werden könnte. 
	\item Moderne Softwareentwicklung verlässt sich auf Komponententests um Seiteneffekte  frühzeitig zu erkennen. Mit automatisierten Tests auf höheren Ebenen werden Nebeneffekte noch weitflächiger erkannt \cite{pol_management_2002}.
	\item Sowohl die Qualitätssicherung als auch die Softwareentwicklung können erheblich entlastet werden. Damit werden Ressourcen frei, die für neue oder verbesserte Funktionalitäten eingesetzt werden können.
\end{itemize}

%=======================================================================
\section{Zielsetzung}
%=======================================================================

\Gls{MBT} ist ein Themengebiet, welches in den letzten 10 bis 15 Jahren zeigte, dass durchaus gute und kostenwirksame Ergebnisse damit erzielt werden können \cite{utting_practical_2007}. Diese Arbeit soll den aktuellen Stand von \Gls{MBT} feststellen und konkret die folgenden Fragestellungen beantworten:

\begin{itemize}
	\item Wie kann \Gls{MBT} in der gegenwärtigen Softwareentwicklung eingesetzt werden und für welche Einsatzzwecke bestehen ausgereifte Technologien?
	\item Kann ein Modellformat und eine passendes Werkzeug gefunden bzw. entwickelt werden, um Testfällle generieren zu lassen, die denselben Nutzen wie manuelle Systemtests haben?
	\item Wie zeit- und ressourcenintensiv ist die Erstellung und Wartung dieser Modelle?
	\item Wie viel exaktes Know-How über die zu testende Applikation ist notwendig, um ein solches Modell zu erstellen? Welche Rollen im Software Team können die Erstellung und Wartung übernehmen?
\end{itemize}

Die methodische Vorgehensweise dieser Arbeit ist in einen Theorie- und einen Praxisteil gegliedert. Der theoretische Teil befasst sich mit dem aktuellen Stand von \Gls{MBT}. Vielversprechende und zeitgemäße Konzepte und Werkzeuge werden erklärt und auf ihre Eignung, in großen agilen Softwareprojekte eingesetzt zu werden, geprüft. Dabei werden Werkzeuge und Vorgehensweise für verschiedene Teststufen in Betracht gezogen.\\
Der praktische Teil besteh darin, eine dieser Methoden in einer Fallstudie im industriellen Umfeld zu erproben. Die Technologie kann in einem lebendigen und komplexen Umfeld angewendet und dementsprechend angepasst werden. Dies garantiert Praxisnähe und erhöht die Qualität der Ergebnisse.

%=======================================================================
\section{Vorgehensweise und Aufbau der Arbeit}
%=======================================================================
Kapitel \ref{sec:fundamentals} \Newnameref{sec:fundamentals} besteht aus zwei Teilen. Erst beschreibt Abschnitt \ref{sec:software_quality} die Grundlagen der Software Qualitätssicherung und umfasst viele Definitionen, die relevant für die Evaluation der Methodiken und Werkzeuge in den späteren Kapiteln sind. Dann folgt Abschnitt \ref{sec:mbt}, der den Begriff \Gls{MBT} im Detail erklärt, die theoretischen Vorteile dieses Ansatzes erläutert und den allgemeinen \Gls{MBT} Testprozess beschreibt. Außerdem wird ein Leitfaden, basierend auf Erkenntnissen mehrerer Quellen, für den Einsatz von \Gls{MBT} in Softwareprojekten geboten.\\

Das Kapitel \ref{sec:problemdescription} \Newnameref{sec:problemdescription} beginnt mit der Beschreibung des Softwareprojekts, welches dem Autor als Fallstudie für diese Arbeit zur Verfügung stand. Es wird erklärt, welche Maßnahmen während der Entwicklung dieses Softwareprojekts getroffen werden, um Qualität sicherzustellen. Basierend auf den erkannten Schwachstellen werden im Abschnitt \ref{sec:modeljunit} und \ref{sec:mbt_integration} Werkzeuge und Methodiken beschrieben, die auf Komponenten- und Integrationstestebene eingesetzt werden können.\\

In Kapitel \ref{sec:results} wird ein umfassendes Konzept beschrieben, um modellbasiertes Testen in großen agilen Softwareprojekten einzusetzen. Im Abschnitt \ref{sec:details} werden die einzelnen Bestandteile dieses Konzepts anhand von Code- und Modellbeispielen erklärt.\\

In Kapitel \ref{sec:mbt_bdt} wird eine transparente Evaluation der, im vorangegangenen Kapitel beschriebenen, neuartigen Teststrategie gemacht. Diese Evaluation richtet sich nach den Erkenntnissen aus der Fallstudie.\\    

Das Diskussionskapitel \ref{sec:discussion} vergleicht diese Arbeit mit anderen dieses Feldes und Abschnitt \ref{sec:discussion_format} erläutert eine der größten Schwächen von \Gls{MBT}. Abschließend fasst Abschnitt \ref{sec:conclusion} die wesentlichen Erkenntnisse zusammen und gibt einen Ausblick in die Zukunft.
