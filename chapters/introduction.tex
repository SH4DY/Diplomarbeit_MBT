%%%%%%%%%%%%%%%%%%%%%%%%%%%%%%%%%%%%%%%%%%%%%%%%%%%%%%%%%%%%%%%%%%%%%%%%
\chapter{Einleitung}
\label{sec:introduction}
%%%%%%%%%%%%%%%%%%%%%%%%%%%%%%%%%%%%%%%%%%%%%%%%%%%%%%%%%%%%%%%%%%%%%%%%

\makeatletter
\ifthesis@masterthesis



Das hier zur Verfügung gestellte Template ist als Hilfestellung gedacht, sie stellt keine verbindliche Richtlinie dar. Der Aufbau einer Diplomarbeit hängt sehr von der bearbeiteten Thematik ab. Diese Vorlage ist für eine Arbeit erstellt, die einen experimentellen Teil enthält (Fallbeispiel). Bei theoretischen Arbeiten oder Programmierungen sind entsprechende Anpassungen vorzunehmen, bitte Rücksprache mit Ihrem Betreuer halten. Bei der Gliederung der Arbeit ist darauf zu achten, dass das Inhaltsverzeichnis einen ersten Eindruck von der thematischen Vollständigkeit sowie der Ausgewogenheit der Behandlung des Themas bietet. Die Gliederung und somit auch der Wahl der Kapitelüberschriften vermitteln einen Eindruck über die angewendete Systematik, die Vollständigkeit der Behandlung des Themas und deren \enquote{Wissenschaftlichkeit}.
\fi
\makeatother

In der Einleitung soll die Zielsetzung der Arbeit beschrieben, ihre Einordnung in einen übergeordneten Kontext hergestellt und die Bedeutung des Themas erörtert werden. Zu diesem Zweck ist die Einleitung in folgende Unterkapitel unterteilt:
\begin{itemize}
	\item Problemstellung
	\item Motivation
	\item Zielsetzung
	\item Aufbau der Arbeit
\end{itemize}

\makeatletter
\ifthesis@masterthesis
Durch die Einleitung sollen folgende Punkte in den jeweiligen Unterkapiteln klargestellt werden:
\begin{itemize}
	\item Etwaige thematische Einschränkungen bzw. Auswahl und Begründung der Bearbeitungsziele
	\item Betrachtung verschiedener methodischer Alternativen zur Aufgabenlösung und Erklärung der Entscheidung
	\item Gewählter Lösungsansatz, z.B. theoretische Untersuchung, Literaturauswertung und -vergleich oder eine empirische, auf eigenen Erhebungen basierende Untersuchung
\end{itemize}
\fi
\makeatother

\paragraph{Organisatorisches}
\makeatletter\ifthesis@masterthesis
Bitte einen Zeitplan für die Verfassung Ihrer Diplomarbeit erarbeiten und mit Ihrem Betreuer abstimmen. Es gibt sehr \enquote{kurzlebige} Themenstellungen, die rasch an Aktualität verlieren, da ist der Zeitplan unbedingt einzuhalten, ansonsten wird das Thema neu vergeben. Bei Themen, die länger aktuell bleiben, kann der Zeitrahmen auch länger erstreckt werden, dies aber bitte im Vorfeld abklären! Üblicherweise sollte die Verfassung einer Diplomarbeit ein halbes Jahr bis ein Jahr dauern, Ausnahmen bitte zumindest durch regelmäßigen email-Kontakt abstimmen.
\fi\makeatother

Der Umfang einer Diplomarbeit beträgt üblicherweise 90 bis ca. 120 Seiten und bei Bachelorarbeiten 40 bis ca. 60 Seiten. Beurteilungskriterien für eine Diplomarbeit ist nicht nur die Qualität der praktischen Arbeit, sondern auch Aufbau, Inhalt und Formulierung der schriftlichen Arbeit. Insbesondere sind die Grundregeln wissenschaftlichen Arbeitens (z.B. richtiges Zitieren) zu beachten.


%=======================================================================
\section{Problemstellung}
%=======================================================================

Steigende Komplexität in modernen technischen Disziplinen, darunter auch die Softwareentwicklung, erfordern Methoden zur Qualitätssicherung. Systematisches Prüfen ist ein zentrales Mittel, um Qualität von Software über einen längeren Zeitraum hinweg sicherzustellen\cite{spillner_software_2014}. Automatisierte Softwaretests auf Integrations- und Systemebene haben sich als effizienter Kontrollmechanismus etabliert\cite{dustin_software_2000}, sind aber herausfordernd in ihrer Umsetzung, vor allem in agilen Entwicklungsumgebungen. Kurze Iterationen, kombiniert mit einer hohen Frequenz von Änderungen in der Software erfordern spezialisierte Methoden\cite{linz_testing_2014}. Allein die Wahl des richtigen Werkzeugs für automatisiertes Testen garantiert noch keine erfolgreiche Teststrategie. Ungeachtet der gewählten Teststrategie, kann das Warten von Testfällen einen erheblichen Ressourcenaufwand verursachen der über Erfolg und Misserfolg eines Softwareprojekts entscheiden kann\cite{dustin_software_2000}. Vor allem agile Teams arbeiten auf multifunktionaler Basis. Deshalb darf sich auch die Generierung und Wartung von Testfällen nicht nur den technischen Rollen erschließen\cite{linz_testing_2014}. Bei mehreren weitverbreiteten Technologien ergibt sich aber genau diese Problematik.\\
Ein vielversprechender Ansatz ist Testen auf Basis von Modellen, sogenannten modellbasiertes Testen. Beim modellbasierten Testen wird die zu testende Applikation als Modell in abstrahierter Form dargestellt. Daraus werden Testfälle automatisch generiert.

%=======================================================================
\section{Motivation}
%=======================================================================
Große Softwareprojekte sind teuer in der Entwicklung und der Wartung. Sie müssen unter hohem Zeitdruck geplant, spezifiziert, implementiert und ausgeliefert werden. Außerdem sollen solche Systeme dann mehrere Jahre gewartet und in ihrer Funktionalität erweitert werden. Softwaresysteme die sensible Daten verarbeiten (Finanzwesen, Gesundheitswesen etc.) sind davon abhängig sehr exakte Ergebnisse zu liefern. Strenges Prüfen dieser ist unumgänglich. Gleichzeitig verknappen solche Tätigkeiten den engen zeitlichen Rahmen solcher Projekte. Deshalb wird, unabhängig von der Branche, das automatisierte regressive Testen von Software thematisiert.

In diesem Kapitel wird der Forschungsbedarf aufgezeigt. Nach dem Lesen dieses Kapitels sollten folgende Punkte klar dargestellt sein:
\begin{itemize}
	\item Aktueller Stand der Wissenschaft in Bezug auf die zuvor formulierte Problemstellung und klare Darstellung, was hier unzureichend/offen ist.
	\item GGf. Darstellung des größeren Forschungsbereichs, in den die Diplomarbeit eingebettet ist.
	\item Darlegung der Bedeutung des Themas für den Stand oder die Weiterentwicklung eines Bereichs der Informatik (z.B. Datenbanksysteme, Mobile Anwendungen, Java-Programmierung, Rechenzentrumsbetrieb, \dots) oder eines Fachbereichs (z.B. Bankwesen, Wertpapierhandel, Gesundheitswesen, Transportwesen, Flugsicherheit \dots). Erklärung, was durch die Lösung des Problems z.B. kostengünstiger/schneller/hochwertiger/sicherer/anwendbarer/\enquote{schöner} etc. wird.
\end{itemize}

%=======================================================================
\section{Zielsetzung}
%=======================================================================

Nachdem die Problemstellung und die Motivation zu deren Lösung formuliert wurden, wird in diesem Kapitel das zu erarbeitende Resultat beschrieben. \todo{Hier weiter...}

\makeatletter\ifthesis@masterthesis
Nach dem Lesen dieses Kapitels sollten folgende Punkte klar dargestellt sein:
\begin{itemize}
	\item Umfang, in dem die Problemstellung am Ende der Arbeit gelöst sein soll bzw. mit welchen Einschränkungen.
	\item Methode zur Erarbeitung des Resultats.
	\item Gibt es einen Theorie- und einen Praxisteil?
	\item Schwerpunkte des Praxisteils (z.B. Durchführung einer Umfrage, Programmierung, Herstellung von Hardware, Erprobung einer Methode in einem konkreten Projekt)?
	\item Art des Resultats (z.B. ein Programm, eine Formel, eine Methode, die Erweiterung einer existierenden Methode, ein Konzept, ein Framework, Hardware-Prototyp, eine bewiesene Erkenntnis)?
\end{itemize}
\fi\makeatother

\makeatletter\ifthesis@masterthesis
%=======================================================================
\section{Aufbau der Arbeit}
%=======================================================================

Beispielhaft:

Kapitel \ref{sec:fundamentals} behandelt sowohl Grundlagen als auch Definitionen und bietet einen Überblick \dots, die als Basis für \dots dienen.

\dots, wird in Kapitel 3 erläutert..

Ein Anwendungsszenario (Fallbeispiel), das \dots, ist in Kapitel 4 dargestellt. Dieses Szenario umfasst \dots.

Kapitel 5 setzt sich \dots. auseinander.

Einsatzmöglichkeiten in der Praxis werden in Kapitel 6 diskutiert.

Abschließend fasst Kapitel \ref{sec:conclusion} die wesentlichen Erkenntnisse zusammen und gibt einen Ausblick in die Zukunft.
\fi\makeatother

%%%%%%%%%%%%%%%%%%%%%%%%%%%%%%%%%%%%%%%%%%%%%%%%%%%%%%%%%%%%%%%%%%%%%%%%%%%%%%%%%%%%%%%%%%%%%%%%%%%%%%%%%%%%%%%%%%%%%%%%%%%%%%%%%%%%%%%%%%%%%%%%%%%%%%%%%%%%%%%%%%%%%%%%%%
% \section{General Information}
%
% This document is intended as a template and guideline and should support the author in the course of doing the master's thesis.
% Assessment criteria comprise the quality of the theoretical and/or practical work as well as structure, content and wording of the written master's thesis. Careful attention should be given to the basics of scientific work (e.g., correct citation).\footnote{Sample Footnote}
%
% \section{Organizational Issues}
%
% A master's thesis at the Faculty of Informatics has to be finished within six months. During this period regular meetings between the advisor(s) and the author have to take place.
% In addition, the following milestones have to be fulfilled:
% \begin{enumerate}
%   \item  Within one month after having fixed the topic of the thesis the master's thesis proposal has to be prepared and must be accepted by the advisor(s). The master's thesis proposal must follow the respective template of the dean of academic affairs. Thereafter the proposal has to be applied for at the deanery. The necessary forms may be found on the web site of the Faculty of Informatics. \url{http://www.informatik.tuwien.ac.at/dekanat/formulare.html}
%   \item  Accompanied with the master's thesis proposal, the structure of the thesis in terms of a table of contents has to be provided.
%   \item Then, the first talk has to be given at the so-called ``Seminar for Master Students''. The slides have to be discussed with the advisor(s) one week in advance. Attendance of the ``Seminar for Master Students'' is compulsory and offers the opportunity to discuss arising problems among other master students.
%   \item At the latest five months after the beginning, a provisional final version of the thesis has to be handed over to the advisor(s).
%   \item As soon as the provisional final version exists, a first poster draft has to be made. The making of a poster is a compulsory part of the ``Seminar for Master Students'' for all master studies at the Faculty of Informatics. Drafts and design guidelines can be found at \url{http://www.informatik.tuwien.ac.at/studium/richtlinien}.
%   \item After having consulted the advisor(s) the second talk has to be held at the ``Seminar for Master Students''.
%   \item At the latest six months after the beginning, the corrected version of the master's thesis and the poster have to be handed over to the advisor(s).
%   \item After completion the master's thesis has to be presented at the ``epilog''. For detailed information on the epilog see: \\ \url{http://www.informatik.tuwien.ac.at/studium/epilog}
% \end{enumerate}
%
% \section{Structure of the Master's Thesis}
%
% If the curriculum regulates the language of the master's thesis to be English (like for ``Business Informatics''), the thesis has to be written in English. Otherwise, the master's thesis may be written in English or in German. The structure of the thesis is predetermined.
% The table of contents is followed by the introduction and the main part, which can vary according to the content. The master's thesis ends with the bibliography (compulsory) and the appendix (optional).
%
% \begin{itemize}
%   \item	Cover page
%   \item Acknowledgements
%   \item Abstract of the thesis in English and German
%   \item Table of contents
%   \item Introduction
%   	\begin{itemize}
%   		\item motivation
%   		\item problem statement (which problem should be solved?)
%   		\item aim of the work
%   		\item methodological approach
%   		\item structure of the work
%   	\end{itemize}
%   \item State of the art / analysis of existing approaches
%   	\begin{itemize}
%   		\item literature studies
%   		\item analysis
%   		\item comparison and summary of existing approaches
%   	\end{itemize}
%   \item Methodology
%   	\begin{itemize}
%   		\item used concepts
%   		\item methods and/or models
%   		\item languages
%   		\item design methods
%   		\item data models
%   		\item analysis methods
%   		\item formalisms
%   	\end{itemize}
%   \item Suggested solution/implementation
%   \item Critical reflection
%   	\begin{itemize}
%   		\item comparison with related work
%   		\item discussion of open issues
%   	\end{itemize}
%   \item Summary and future work
%   \item Appendix: source code, data models, \dots
%   \item Bibliography
% \end{itemize}
%