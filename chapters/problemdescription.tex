%%%%%%%%%%%%%%%%%%%%%%%%%%%%%%%%%%%%%%%%%%%%%%%%%%%%%%%%%%%%%%%%%%%%%%%%
\chapter{Umfeldbeschreibung: Testing von Finanz-Software}
\label{sec:problemdescription}
%%%%%%%%%%%%%%%%%%%%%%%%%%%%%%%%%%%%%%%%%%%%%%%%%%%%%%%%%%%%%%%%%%%%%%%%
\section{Merkmale und Besonderheiten von Software im Bankenumfeld}

\section{Ist-Analyse Partnerunternehmen}

\subsection{Hintergrund zu den Applikation RETo und RESi}
Die Applikation \textit{RESi}, in anderer Struktur und bis in das Jahr 2013 \textit{RETo} (Raiffeisen Expert Tool), ist ein Backend-Service der Dienste für die verschiedensten Front-End Applikationen des Unternehmens bereitstellt. Unter altem Namen, bot die Applikation früher eine eigene Benutzeroberfläche und weniger, dafür spezifischere Funktionalität. RETo war im Intranet der Fillialen im Einsatz und wurde ausschließlich für Kundenberatungen eingesetzt. Die Applikation hatte folgende Kernkompetenzen:

\begin{itemize}
\item Ermittlung von Anlegerprofilen
\item Durchführung von Anlage-Checks
\item Beratungen zum Thema Wohnen und Wohneigentum
\item Beratungen zum Thema Vorsorge
\item Beratungen zum Thema Pension
\end{itemize}

Im Jahr 2014 wurde die ein Großteil der Applikationslandschaft umstrukturiert. In einer Bemühung einzelne Applikationen zu vereinheitlichen und Mehrfachaufwendungen zu minimieren, wurde RETo modularisiert. Das Projekt RESi wurde gestartet. RESi stellt eine Back-End-Service Schicht dar. Da RETo bereits einen sehr breiten Funktionsumfang bot, besteht RESi zu großen Teilen aus den RETo Kernkomponenten. Auch das Entwicklungsteam ist das selbe geblieben. Mehrere, ehemals eigenständige Applikationen, greifen jetzt auf RESi zu und bieten nur noch ein Front-End und eine dünne Serverschicht. RETo existiert weiterhin aber greift auch auf die Services von RESi zu.

\subsection{Entwicklungsumfeld}
Das Umfeld er Applikaiton RESi wird in einem klassischen Auftragnehmer/Auftraggeber Szenario entwickelt. Eine Fachabteilung stellt den Kunden dar. Sie definieren die Anforderungen an die Applikation. Dem gegenüber ist das Entwicklungsteam. Dieses besteht aus einem Applikationsmanager und 6-8 Entwicklern. 

\begin{itemize}
\item 2 Major Release pro Jahr
\item Mehrere Service-Release pro Jahr
\item 3-wöchige Scrum-Sprints
\item 4 Ebenen (Integrationstest, Systemtest, Akzeptanztest, Produktion)
\item Scrum Master wechselt
\item Product Owner (PO) wird von Fachbereich gestellt
\item tägliche Scrum-Meetings
\end{itemize} 

Die Applikation RESi ist, bedingt durch ihre Reife, bereits sehr stabil. Effektiv ist sie schon seit mehreren Jahren (in anderer Struktur und unter anderem Namen) im produktiven Einsatz. Der typische Entwicklungszyklus wird also vom Fachbereich angestoßen. Ein Request For Change (RFC) wird vom Fachbereich angenommen oder verfasst. Die zuständigen Produktmanager definieren genügend fachliche Details, bevor ein RFC in eine Story verwandelt wird. Diese Story fließt nun typischerweise in den Scrum-Backlog (\todo{Ref zu Scrum Kapitel}). Wenn auf einer der Entwicklungsebenen Fehler mit hoher Priorität gefunden werden, wird ein Defect-Bericht verfasst, der direkt in den laufenden Entwicklungszyklus (Sprint) einfließt.\\
Auf Seiten der Entwickler finden kurze tägliche Meetings statt. Dauer und Struktur dieser täglichen Meetings entsprechen den klassischen Daily-Scrum Standups. Neben den Entwicklern ist der Scrum-Master und der Product Owner aus dem Fachbereich anwesend. Er steht für kurzfristig auftretende Fragen bereit. Weiters finden im Entwicklungsteam wöchentliche Research-Meetings statt. In diesen werden die neu eingetroffenen Stories analysiert und modularisiert. Ziel ist es, eine Story in Tasks herunterzubrechen, die in einem Arbeitstag schaffbar sind. Ein Task soll also von einem einzelnen Entwickler implementiert werden. Im selben Zug, wird werden die Aufwände der Story und damit der Tasks geschätzt. Im RESi Team kommen zwei verschiedene Methoden zur Aufwandsschätzung zum Einsatz. Einerseits wird eine Methode verwendet wo alle Stories offen und ungeordnet aufgelegt werden. Nun wird das Team der Reihe nach gebeten, eine Story einzuordnen. Damit werden weder Storypoints noch Stunden geschätzt. Stories werden anhand ihrer augenscheinlichen Aufwänden geordnet. Wenn kein Teammitglied mehr eine Änderung machen will, endet die Aufwandsschätzung. Der Scrum-Master legt schlussendlich, in Absprache mit dem Team, fest welcher Bereich von Stories einen zukünftigen Sprint fließen. Nimmt man Stories aus dem vorderen Bereich der Reihung, mindert dies den Gesamtaufwand viel höher als wenn Stories aus dem hinteren Teil zurück in den Backlog verschoben werden. 

\subsubsection{Entwicklungsebenen}
Zwischen der Implementierung eines Tasks und dessen Eintritt in eine produktive Umgebung, läuft dieser durch definierte Ebenen. Während der Bearbeitung eines Tasks, benutzen die Entwickler eine lokale Installation der Applikation (diese entspricht der Ebene \textit{Integrationstest}). Wenn ein Entwickler einen Task abschließt und alle Unit-Tests erwartungsgemäß durchlaufen werden, wird der Task zum \textit{Code Review} freigegeben. Ein anderer Entwickler liest den Code und gibt Feedback zu Richtigkeit, Effizienz, Lesbarkeit und Stil des Code-Stücks. Das Code-Stück wird, nach eventueller Korrektur, auf der Ebene \textit{Integrationstest} deployed (ein IBM WebSphere Applikationsserver \footnote{IBM WebSphere \url{http://www.ibm.com/websphere}} im Falle von RESi). Diese Ebene hat die Hauptaufgabe, auftretende Nebeneffekte aufzudecken, die das neu programmierte Code-Stück verursacht. Ab diesem Zeitpunkt beginnt der Fachbereich bereits mit manuellen Tests auf dem Integrationsserver. In manchen Fällen macht es keinen Sinn gegen einen einzelnen Task zu testen (möglicherweise lässt er sich GUI-seitig auch gar nicht testen). Dann wird abgewartet bis sich verwandte Tasks oder die ganze zugehörige Story für den Integrationsserver freigegeben werden. Da Entwickler und Fachbereich täglich auf dieser Ebene arbeiten, werden auftretende Nebeneffekte durch Wartungsänderungen eher entdeckt.\\
Nach Abschluss eines Sprints wird der Stand der Ebene \textit{Integrationstest} auf \textit{Systemtest} deployed. Hier testet der Fachbereich genau definierte Abläufe. Außerdem unternimmt eine gesonderte Test-Abteilung Last- und Performance Tests auf dieser Ebene. Bis zu diesem Zeitpunkt läuft die Entwicklung relativ streng nach agilen Prinzipien. Um das Zusammenspiel der Applikationslandschaft zu vereinheitlichen, wird unternehmensweit aber immer noch auf viel längere Release-Zyklen gesetzt. Sprint-Ergebnisse werden also nicht zeitnah in die Produktionsebene versetzt. Die Ebene \textit{Akzeptanztest} wird also zwischen Systemtest und \textit{Produktionsebene} gezogen. Auf ihr werden die Stände getestet, die für Major-Releases geplant sind. Typischerweise wird gegen Ende eines Major-Release Zyklus verstärkt auf der Ebene \textit{Akzeptanztest} deployed und getestet. Trotzdem kann es zu Überlappungen kommen die mit dem agilen Iterationszyklus interferieren. Sprint-Ergebnisse die kurz vor einem Major-Release eigentlich für produktionsreife getestet werden sollten, werden möglicherweise erst für die nächste Veröffentlichung beachtet. Erstens werden also Testressourcen periodisch für \textit{Akzeptanztest} benötigt, obwohl bereits neuere Versionen der Applikation testbar wären. Zweitens leidet das Endprodukt wenn Features, die eigentlich zur Veröffentlichung freigegeben werden könnten, mehrmonatige Verspätungen haben.

\section{Qualitätssicherung im Projekt}
\subsection{Versuch der Qualitätssicherung durch skriptgesteuerte GUI-Tests}
\subsection{Abhilfe durch Modellbasiertes Testen der Schnittstellen?}