\makeatletter\ifthesis@masterthesis
%%%%%%%%%%%%%%%%%%%%%%%%%%%%%%%%%%%%%%%%%%%%%%%%%%%%%%%%%%%%%%%%%%%%%%%%
\chapter{Diskussion \& Zusammenfassung}
\label{sec:discussion}
%%%%%%%%%%%%%%%%%%%%%%%%%%%%%%%%%%%%%%%%%%%%%%%%%%%%%%%%%%%%%%%%%%%%%%%%

\section{Vergleich mit verwandten Arbeiten}
\todo[color=yellow, inline]{@Boris, stark überarbeitet}
Im Laufe der Recherche für diese Arbeit hat sich gezeigt, dass modellbasiertes Testen ein sehr umfangreiches Thema ist. Es finden sich Arbeiten, Fallstudien und Anwendungsbeispiele in der Industrie, die zwar alle von modellbasiertem Testen sprechen, aber sich trotzdem gravierend unterscheiden. Deshalb ist ein Vergleich all dieser Methoden und Werkzeuge schwierig, unterstreicht aber gleichzeitig die Wichtigkeit von modellbasiertem Testen auf die Struktur des Projekts und die Art der zu testenden Software maßzuschneidern\todo{zu langer Satz}. Weiters kann \Gls{MBT} in allen bekannten Teststufen (siehe Abschnitt \ref{sec:teststufen} für eine Beschreibung der Stufen in der Domäne des Softwaretests) eingesetzt werden und abhängig von der Stärke\todo{anderes Wort} der bestehenden Testqualität macht es Sinn \Gls{MBT} nur punktuell einzusetzen. Mit dem in dieser Arbeit vorgeschlagenen Konzept ist dies, im Gegensatz zu gesamtheitlichen Lösungsansätzen für \Gls{MBT}, ohne weiteres möglich.\\
Viele Arbeiten wie \cite{pretschner_one_2005} \cite{pinheiro_model-based_2013} \cite{sensler_testautomatisierung_2011} präsentieren Konzepte und stellen die verwendeten Werkzeuge und das technische Vorgehen in den Mittelpunkt. Die menschliche Komponente wird dabei nicht erwähnt. In der Fallstudie dieser Arbeit hat sich gezeigt, dass es nicht reicht ein generisches aber technisch ausgereiftes Konzept einzuführen. In umfangreichen Softwareprojekten sind sehr viele Akteure mit den verschiedensten technischen und fachlichen Hintergründen beteiligt. Wenn die Teststrategie nicht diesen Umständen angepasst wird, leidet die Qualität des gesamten Projekts darunter. Ziel einer Teststrategie sollte sein, das Softwaretesten für so viele involvierte Parteien wie möglich verständlich zu machen. Im Sinne des gemeinsamen Vokabulars (die Grundprinzipien von \Gls{BDD} werden in Abschnitt \ref{sec:bdd} erklärt) ist es nur von Vorteil, wenn zwischen fachlichem und technischem Testen volle Transparenz herrscht. Weiters ist die Beteiligung der Akteure, die das zu entwickelnde Stück\todo{piece of Software :-) anderes Wort} Software spezifizieren, wünschenswert. Das Produktdesign\todo{Wieso nicht Qualität?} wird \textit{bewusster} und gleichzeitig präziser. Der zeitliche Aufwand für die Spezifikation erhöht sich möglicherweise, es wurde in der Vergangenheit mehrfach bewiesen, dass Ungenauigkeiten in der Spezifikation massive Auswirkungen auf das Endprodukt haben. Die Kosten zur Beseitigung erhöhen sich drastisch sich mit dem Fortschritt des Projekts. Deshalb ist der erhöhte Zeitaufwand für die Modellierung und Kommunikation in den frühen Stadien des Projekts gut investierte Zeit\todo{besser: wirkt sich positiv auf ... aus}.\\
Einige Werkzeuge\todo{Auflistung}, unter anderem auch das in den Experimenten von IBM \cite{farchi_using_2002} verwendete, setzen voraus, dass in den Modellen gewisse Abdeckungskriterien definiert werden müssen. Meist sind dies Werkzeuge, die textuelle Modellierungssprachen verwenden. Diese haben den Vorteil, dass sehr präzise Vor- und Nachbedingungen (Pre- und Postconditions) definiert werden können. Verglichen mit der Kombination von Graphwalker, mit grafischer Notation, und BDD-Tests sind diese weniger flexibel. Die Notation von Graphwalker ist so simpel gehalten, dass keinerlei Aussagen über Abdeckungs- oder Abbruchkriterien \todo{statt Stoppkriterien?} gemacht werden können. Das Modell und dessen Adapter-Code wird offline generiert (siehe Abschnitt \nameref{sec:online_offline} \ref{sec:online_offline}). Modell und Adaptercode\todo{Unterschiedliche Schreibweise, ganze Arbeit prüfen, besser ohne Bindestrich} können nun integriert oder von beliebiger Stelle aus gestartet werden. Das Modell und dessen Traversierung wird vollkommen entkoppelt. Das Modell muss also nicht dupliziert und modifiziert werden, um verschiedene Traversierungen zu generieren.\\
Grafische Notationen haben weiters den Vorteil, dass sie durch die Visualisierung leichter verständlich sind. Gleichzeitig besteht die Gefahr, dass Details übersehen werden oder gar nicht präzise genug notiert werden können. Das vorgestellte Konzept geht diesen Missstand auf zwei Arten an. Einerseits bietet das gewählte Werkzeuge eine sehr schlanke Syntax, die es nicht zulässt, dass wichtige Details unsichtbar modelliert werden können. Die grafische Syntax bestraft\todo{Wortwahl} den Testentwickler aber auch nicht, indem nur ein Subset von Fähigkeiten in der grafischen Syntax verfügbar gemacht werden. Andererseits, wird einfach auf BDD-Tests zurückgegeriffen wenn sehr spezifische Ein- und Ausgabe Kombinationen auf einem gewissen Zustand geprüft werden sollen. Wie in Abschnitt \ref{mbt_bdt} erklärt, kann durch die Vereinigung von \Gls{BDT} und \Gls{MBT} auf die Expressivität\todo{Wortwahl?} von \Gls{BDT} \textit{Then} Klauseln zurückgegriffen werden, wenn dies am nötigsten ist: In Knoten in denen Überprüfungen stattfinden, die in der grafischen Notation nicht ersichtlich sind.\\
Ein anderer Gesichtspunkt ist die Einfachheit der Einführung der Teststrategie. Generische Plattformen, wie die von \todo{citeauthor tag verwenden!}Zech et al. \cite{zech_generic_2012} vorgeschlagene, konzentrieren sich nicht nur auf die Modellierung und Traversierung des \Gls{SUT}. Vielmehr wollen sie eine gesamtheitliche Basis für das Testen schaffen. In großen Langzeitsoftwareprojekten ist eine solche Einführung aber extrem schwierig und in der Praxis selten. Die vorgeschlagene Teststrategie dieser Arbeit lässt eine inkrementelle und, zu agilen Projekten passende, iterative Einführung zu. Ebenso kann der modellbasierte Teil des Konzepts auf eigenständige Module des SUT's\todo{SUT ohne TAG!} beschränkt werden. Der BDD-Teil des Konzept kann zwar auch iterativ eingeführt werden, aber eine Beschränkung auf Teile des Projekts würde keinen Sinn machen. Dies hätte zur Folge, dass die Spezifikationen in Form von \Gls{COS} mittels zwei verschiedener System festgehalten oder versioniert werden müssten. Mittel- bis langfristig entstünden Überschneidungen und Unklarheiten, welche \Gls{COS} denn nun gültig sind und bei welchen es nötig ist, sie in das Regressionstesten einfließen zu lassen.\\
Robert Binders\todo{Citeauthor Tag!} Ansatz, den er in einer Präsentation \cite{binder_model-based_2014} erklärte aber in keiner schriftlichen Publikation detaillierte, scheint eine sehr ähnliche Idee zu verfolgen. Auch er hebt heraus wie sich \Gls{MBT} und \Gls{BDD} ergänzen und in die agile Vorgehensweise passen. Der Gedanke der gemeinsamen Verwendung von \Gls{BDT} und \Gls{MBT} wurde in dieser Arbeit weitergeführt. Daraus entstanden ist nicht nur die Empfehlung zur parallelen Verwendung beider Techniken, sondern auch eine Anleitung zur Vereinigung und Nutzung der jeweiligen Stärken.


\section{Kein Modellformat hat sich etabliert}
\label{sec:discussion_format}
Die Auswahl des Modellformats und die zugrunde liegende Notation ist eine fundamentale Entscheidung bei der Einführung von modellbasiertem Testen. Meist geht die Auswahl eines \Gls{MBT} Werkzeugs mit der Wahl des Modellformats einher. Die Ursprünge dieser Werkzeuge, für welche Art von Softwareprojekten diese zumeist eingesetzt wurden, sind unschwer an der unterstützten Modellnotation zu erkennen.  In der Praxis besteht eine Vielzahl von Notationen und Formaten, die auf sehr unterschiedlichen Prinzipien beruhen. Eine Transformation von einem Modellformat in ein anderes ist kaum möglich (beispielsweise lässt sich ein Modell in B Notation nur mit genauer Kenntnis des \Gls{SUT} in die Graphwalker-Notation konvertieren.)\\
Erschwerend kommt hinzu, dass gerade die großen kommerziellen Werkzeuge (wie Leiros Smartesting \footnote{Leiros Smartesting Produktseite \url{http://www.smartesting.com/en/}} oder Conformiq's \Gls{MBT} Toolsuite \footnote{Conformiq Produktseite \url{https://www.conformiq.com/products/}}) vollständig auf nicht-portable Eigenentwicklungen setzen. Diese Hersteller versuchen gezielt große Software-Projekte in ihr Ökosystem einzusperren, verpassen es aber die Branche der \Gls{MBT} Werkzeuge voranzutreiben. Sie vermarkten ihre Modellformate als Produktgeheimnis und verhindern dadurch die Entwicklung eines Standards der modellbasiertes Testen viel stärker in Erscheinung treten ließe. Weiters lassen sich Fähigkeiten der Werkzeuge kaum vergleichen. Während die Firma Conformiq ohne weiteres Testversionen ihrer vollständigen Toolsuite anbietet und die Open-Source Werkzeuge ohnehin frei evaluierbar sind, verlangt die Firma Leiros knapp 1000 Euro für eine akademische Lizenz. Auch auf Nachfrage des Autors wurde keine Demo- oder Testversion zum Zweck dieser Diplomarbeit zur Verfügung gestellt. Anhand offizieller Produktbeschreibungen und Quellen lässt sich nicht herausfinden wie sich der Workflow mit dem genannten Programm verhält, geschweige denn wie die zugrunde liegende Modellierung funktioniert.\\
Die Bemühungen der \Gls{OMG} mit dem UML Testing Profile eine einheitliche Sprache für modellbasiertes Testen zu schaffen ist ebenfalls nicht gelungen. Die Zahl der Publikationen die sich auf das UML Testing Profile stützen ist seit 2007 zwar konstant \footnote{Nach Recherche auf einschlägigen Portalen wie \url{http://ieeexplore.ieee.org/}}, trotzdem bietet kein dem Autor bekanntes Werkzeug Unterstützung dafür an. Zwar schlägt die offizielle Publikation der OMG\todo{TAG oder eintrag im Glossar} \cite{_model-driven_2007} manuelle Vorgehensweisen vor um UTP\todo{UTP Tag/Glossar} zu benutzen (siehe Kapitel \ref{sec:utp}), dies ist im Umfeld großer agiler Softwareprojekte aber nicht praktikabel. Zu viele Fehler könnten bei der manuellen Transformation von UTP-Diagramme\todo{UTP generell auf TAG prüfen} passieren.\\
Im Kontext von skriptbasierten GUI-Tests hat sich gezeigt, dass es mittel- bis langfristig enorme Vorteile mit sich bringt unabhängig von einem Hersteller zu bleiben. Diese Fallstudie war ebenfalls in der Finanzbranche angesiedelt \cite{graham_experiences_2012}. Die Fallstudie dieser Arbeit (siehe Abschnitt \ref{sec:fallstudie} \nameref{sec:fallstudie}) hat gezeigt, dass diese Erkenntnis auch im Umfeld des modellbasierten Testens gilt. Die Verwendung eines nicht-proprietären Modellformats bringt, abgesehen von der Flexibilität und Ungebundenheit, mehrere Vorteile:

\begin{itemize}
\item \textbf{Portabilität und Lesbarkeit:} Sollte ein Wechsel der Entwicklungsplattformen oder sogar der Teststrategie bevorstehen, können Modelle, die in einem offenen Format wie XML serialisiert werden können, mühelos transformiert werden. Es bleibt aber zu beachten, dass Transformationen zwischen fundamental verschiedenen Modellierparadigmen, trotz offener Serialisierung, kaum machbar sind.
\item \textbf{Wiederverwendbarkeit:} Leichtgewichtige und quelloffene Werkzeuge, wie Graphwalker\todo{verlinkung nur bei erster Erwähnung in dem Kapitel 5, also weiter oben!} (siehe Abschnitt \ref{sec:graphwalker}), erlauben die Weiterbenutzung bestehender skriptbasierter Tests. In der Praxis hat sich außerdem gezeigt, dass es sehr vorteilhaft ist Teile der Testbasis iterativ auf modellbasierte Tests zu übertragen. Der Aufbau der Modellierungskenntnisse im Team geschieht so breitflächiger und leichtgängiger, als wenn eine große Menge von Modellen und Adaptercode auf einmal erstellt werden müssen. Aus Sicht des Managements müssen weniger Aufwände initial riskiert werden, um projektrelevante Erfahrungen mit \Gls{MBT} zu sammeln. Aus Sicht der Entwickler besteht viel Freiraum in der Werkzeugauswahl für Adaptercode.
\item \textbf{Transparenz:} Technisch versierte Tester bzw. Testentwickler schätzen es, wenn das Parsen des Modells und die Testfallgenerierung möglichst transparent passieren. Bei omnipotenten \Gls{MBT} Lösungen ist dies nicht der Fall, da sie diese Details vor dem Benutzer verbergen wollen um attraktiv für die Fachbereiche zu wirken. In der Praxis sind aber Entwickler sowohl in der Auswahl des Werkzeugs als auch in der Erstellung der Modelle stark beteiligt.
\end{itemize}


% \section{Erweiterung des Begriffs `Test Non-Stop'}
% Nicht nur End to End Tests sondern Testen in Produktion\\
% Erstens: Ansatz von Google mit Machine Learning Ansätze

% \section{Priorisierung und Erweiterung von Unit-Tests}
% \label{sec:discussion_unit}
% Unit-Tests müssen noch mehr an Wichtigkeit gewinnen im agilen Umfeld \cite{linz_testing_2014}, auch Google Testing Blog zitieren. Ich schlage eine noch extremere Test-Pyramide vor (unproportional GUTE Unit-Tests, basierend auf MBT).\\
% Sogar GUI Tests können als Unit-Test angesehen werden, mocken von Daten und Umgebungen --> Verweis zu Service-Virtualisierungen in großen Tools wie HP UFT\\
% Klarstellen, dass End2End Systemtests extrem zeitraubende Unterfangen sind aber gleichzeitig sehr schnell gebrochen werden können (durch Technologieänderungen, Umstellungen der GUU/Plattform usw).

%%%%%%%%%%%%%%%%%%%%%%%%%%%%%%%%%%%%%%%%%%%%%%%%%%%%
\section{Zusammenfassung und Ausblick}
\label{sec:conclusion}
%%%%%%%%%%%%%%%%%%%%%%%%%%%%%%%%%%%%%%%%%%%%%%%%%%%%%%%%%%%%%%%%%%%%%%%%
Die Komplexität moderner Softwaresysteme ist bereits sehr hoch und wird mit wachsender Rechenleistung und Konnektivität noch höher. Es braucht Strategien, um der steigenden Komplexität in der Qualitätssicherung gerecht zu werden. Das Feld der modellbasierten Testtechnologien ist vielfältig. Diese Arbeit zeigt, dass \Gls{MBT} automatische Tests weiter stärken kann, nicht nur um wiederkehrende und sich wiederholende Tätigkeiten zu testen. Auch die vollständige Exploration der Programmzweige und die damit verbundene Gewissheit ermöglicht \Gls{MBT} so einfach wie keine andere Testmethodik.\\
Im Kapitel \ref{sec:problemdescription} \nameref{sec:problemdescription} wurden \Gls{MBT} Werkzeuge für den Einsatz auf Ebene des Komponententests und des Integrationstests erprobt. Das Werkzeug \textit{ModelJUnit} ist in der Verwendung an die weit verbreiteten Komponententestwerkzeuge angelehnt. Modelle werden als Java-Klasse dargestellt und Zustandsübergänge werden auf Methodenebene annotiert. Das Werkzeug bietet keine grafische Notation und erzeugt auch keine visuelle Darstellung. Die Stärken von ModelJUnit liegen in der Einfachheit und in der guten Integrierbarkeit in bestehende Strukturen. Nachteilig wirkt sich aus, dass die Modelle weder die Verständlichkeit von grafischen Modellen, noch die Ausdrucksstärke von textuellen Notationen besitzt.\\
Auf Integrationstestebene gab es Bemühungen, den bekannten UML-Standard auszubauen und für die Definition von Testmodellen verwenden zu können. Das Ergebnis ist das UML Testing Profile. Dabei wurde die Sprache UML durch das Konzept der Profile erweitert und bietet nun Notationsmöglichkeiten speziell für das Testen. UTP\todo{UTP ohne TAG, check des ganzen discussion.tex file!} bietet zwar einige essentielle Konzepte, die für den Gebrauch in komplexen Softwareprojekten nötig sind, aber wie die Generierung der Modelle erfolgen soll, bleibt den Werkzeugentwicklern vorbehalten. Eine Möglichkeit ist das manuelle Mapping von UTP Elementen zu JUnit, wie in Kapitel \ref{sec:utp} beschrieben. Dies gestaltet sich allerdings als recht aufwendig und bietet wenig zusätzlichen Nutzen, verglichen mit der direkten Erstellung der JUnit Testfälle ohne den Umweg über UTP.\\
Auch das in Kapitel \ref{sec:graphwalker} beschriebene Werkzeug \textit{Graphwalker} lässt sich auf Integrationstestebene einsetzen. Es bietet einen Parser, der das Einlesen von Modellen in Form von Graphen ermöglicht. Graphwalker generiert daraus Java-Code, der mit Code für die Verbindung zum \Gls{SUT} gefüllt werden muss. Schlussendlich traviersiert Graphwalker das Modell auf die gewünschte Art und Weise. Das Werkzeug befähigt den Tester modellbasierte Tests gegen das \Gls{SUT} auszuführen, ohne auf gewohnte Mittel zur Verbindungsherstellung mit dem \Gls{SUT} verzichten zu müssen. Schnittstellentests könnten beispielsweise weiterhin mit \textit{SoapUI}, Oberflächentests mit \textit{Selenium} realisiert werden.\\
Die Fallstudie \ref{sec:fallstudie} \nameref{sec:fallstudie} hat gezeigt, dass modellbasierte Tests sinnvoll in agilen Softwareprojekten eingesetzt werden können, auch ohne damit andere Testmethodiken zu verdrängen. Sogar eine schrittweise Einführung in ein großes bestehendes Softwareprojekt ist möglich, vorausgesetzt es werden flexible Werkzeuge genutzt. Der Einsatz von Graphwalker hat sich als sehr unkompliziert herausgestellt, weil wenige tiefgreifende Eingriffe in die bestehende Teststrategie gemacht werden müssen. Das Werkzeug kann unabhängig von der gegebenen Umgebung genutzt werden. Es integriert sich, wenn gewünscht, auch in große Testmanagementprogramme von HP oder IBM. Dies ist durch die offene Java Schnittstelle möglich, die Graphwalker mitbringt. Außerdem ist es nicht nötig bestehenden Adapter-Code neu zu schreiben. Testfälle die als Java-Code vorliegen (z.B. Selenium) oder zumindest durch eine Java Schnittstelle ansprechbar sind (z.B. SoapUI) können in Graphwalker Tests eingebunden werden, weil Graphwalker selbst keine Annahme über das Ansprechen des \Gls{SUT} trifft.\\
%Schreiben wie die Kombination von BDD den Fachbereich einbindet und seine punktuellen Eigenschaften verliert
Das Resultatskapitel \ref{sec:results} hat eine Teststrategie vorgestellt, die auf die Schwächen bestehender Lösungen eingeht. Mittels der durchdachten Orchestrierung mehrerer Testansätze (gewöhnliche Komponententests, \Gls{BDT} und \Gls{MBT}) kann hocheffiziente Qualitätssicherung realisiert werden. Durch die Nutzung einer gemeinsamen Basis für Adapter-Code (der Testing-API, siehe Abschnitt \ref{sec:testing_api}) kann kurzfristig Wartungsaufwand minimiert werden und langfristig Zukunftssicherheit bei einem Werkzeugwechsel geboten werden. 
Abhängig vom fachlichen Kontext, können aufbauend darauf \Gls{BDT} und \Gls{MBT} Tests parallel zueinander eingesetzt werden. Beide Methodiken können dort verwendet werden, wo sie am besten passen, ohne damit den Wartungsaufwand zu erhöhen, weil sie beide von einer gemeinsamen Schicht konsumieren\todo{formulierung konsumieren}.\\

Es hat sich außerdem gezeigt, dass eine Verschmelzung von \Gls{BDT} und \Gls{MBT} sinnvoll sein kann. Die Techniken haben strukturelle Ähnlichkeiten, wenn man betrachtet wie Tests definiert werden und welche Sicht auf das \Gls{SUT} sie einnehmen. \Gls{BDT} definiert mit seinen \textit{Given} und \textit{When} Teilen eine Aktion, die auf einem System in einem bestimmten Zustand ausgeführt wird. Der \textit{Then} Teil ist eine Aneinanderreihung von Überprüfungen, die zusammen evaluiert und über das Resultat des Testfalls entscheiden. Sieht man das \Gls{SUT} als abstrahierten endlichen Automaten und visualisiert man einen solchen \textit{Given-When-Then} Block, dann ist ein \Gls{BDT} Test ein Pfad von einem Knoten zu einem anderen. Sowohl die beiden Notationen, als auch die darunterliegenden Technologien haben ihre Stärken und Schwächen. Der \textit{Given} Teil ist eine kurze Zusammenfassung des Systemzustandes und kann als Modell genauer ausgedrückt werden. Wenn der Zustand aus dem Kontext klar ist, kann es aber genauso valide sein den knappen Einzeiler des \textit{Given} Teil zu verwenden und mit nur einem Knoten im Modell zu starten, der diesen repräsentiert. Der \textit{When} Teil kann fälschlicherweise suggerieren, dass eine Aktion eine atomare Handlung ist oder keine Alternativpfade zum Endzustand bestehen. In einem Modell kann durch mehrere Äste ausgeführt werden, auf welche Arten der Endzustand erreicht werden kann. Auf der Benutzeroberfläche wäre beispielsweise denkbar, dass eine Maske über einen Button, ein Kontextmenü oder ein Tastaturkürzel erreicht werden kann. Zuletzt steht der \textit{Then} Teil, der sich in einem Modell nicht von anderen Knoten unterscheidet. In der Realität finden dort aber die wichtigsten Überprüfungen statt. Diese verbergen sich im Code und sind auf grafischer Ebene nicht sichtbar. Viel ausdrücklicher ist hier \Gls{BDT} indem die Erfolgsbedingungen klar ausformuliert werden. Eine Ebene über einzelnen Tests können auch \Gls{BDT}-Szenarien als Modell dargestellt werden und wiederum mit den ausdrucksstarken \textit{Then}\todo{Vorsicht Schreibweise, alle mit Anfangsbuchstaben groß (Kontrolle Arbeit!)} Statements beendet werden.\\


Für die weitere Festigung dieses Themengebiets wären mehrjährige Fallstudien großer und lange andauernder Softwareprojekte höchst interessant. Durch den Vergleich von Qualitätsmetriken, aufgenommen während modellbasierte und nicht-modellbasierte Werkzeuge im Einsatz waren, ließen sich Schlüsse auf die tatsächliche Produktivitäts- und Qualitätssteigerung schließen die \Gls{MBT} verspricht.\\
Weiters bedarf es einer gewissen Standardisierung von Modellformaten und Werkzeugschnittstellen (siehe die Diskussion in Abschnitt \ref{sec:discussion_format}). Gegenwärtig werden Unternehmen eher abgeschreckt, Ressourcen in eine modellbasierte Strategie zu investieren für die keine entsprechende Standardisierung der Sprache vorliegt. Mit dem Einstellen\todo{Formulierung, verstehe nicht ganz was Du damit meinst} eines kritischen Werkzeugs könnten langjährige Bemühungen auf einen Schlag obsolet werden. Dieser Umstand kann durch die Möglichkeit Testartefakten mühelos wiederverwenden zu können, gemildert werden. Das wiederum ist zur Zeit nicht möglich, da die vielen erhältlichen Werkzeuge grundlegend verschiedene Ansätze verfolgen.



























