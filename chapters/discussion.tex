\makeatletter\ifthesis@masterthesis
%%%%%%%%%%%%%%%%%%%%%%%%%%%%%%%%%%%%%%%%%%%%%%%%%%%%%%%%%%%%%%%%%%%%%%%%
\chapter{Diskussion}
\label{sec:discussion}
%%%%%%%%%%%%%%%%%%%%%%%%%%%%%%%%%%%%%%%%%%%%%%%%%%%%%%%%%%%%%%%%%%%%%%%%

Den akademischen Wert der Arbeit hervorheben, Vergleich mit verwandten Arbeiten: In welchem Verhältnis stehen die Ergebnisse der Diplomarbeit zu den Ergebnissen anderer Studien? Wo gibt es Unterschiede, wo Gemeinsamkeiten? Warum?

Diskussion offener Punkte, Darstellen der Stärken und Schwächen der vorliegenden Ergebnisse.


\section{Erweiterung des Begriffs `Test Non-Stop'}
Nicht nur End to End Tests sondern Testen in Produktion\\
Erstens: Ansatz von Google mit Machine Learning Ansätze

\section{Priorisierung und Erweiterung von Unit-Tests}
Unit-Tests müssen noch mehr an Wichtigkeit gewinnen im agilen Umfeld \cite{linz_testing_2014}, auch Google Testing Blog zitieren. Ich schlage eine noch extremere Test-Pyramide vor (unproportional GUTE Unit-Tests, basierend auf MBT).\\
Sogar GUI Tests können als Unit-Test angesehen werden, mocken von Daten und Umgebungen --> Verweis zu Service-Virtualisierungen in großen Tools wie HP UFT\\
Klarstellen, dass End2End Systemtests extrem zeitraubende Unterfangen sind aber gleichzeitig sehr schnell gebrochen werden können (durch Technologieänderungen, Umstellungen der GUU/Plattform usw).

\fi\makeatother