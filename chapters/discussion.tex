\makeatletter\ifthesis@masterthesis
%%%%%%%%%%%%%%%%%%%%%%%%%%%%%%%%%%%%%%%%%%%%%%%%%%%%%%%%%%%%%%%%%%%%%%%%
\chapter{Diskussion}
\label{sec:discussion}
%%%%%%%%%%%%%%%%%%%%%%%%%%%%%%%%%%%%%%%%%%%%%%%%%%%%%%%%%%%%%%%%%%%%%%%%

Den akademischen Wert der Arbeit hervorheben, Vergleich mit verwandten Arbeiten: In welchem Verhältnis stehen die Ergebnisse der Diplomarbeit zu den Ergebnissen anderer Studien? Wo gibt es Unterschiede, wo Gemeinsamkeiten? Warum?

Diskussion offener Punkte, Darstellen der Stärken und Schwächen der vorliegenden Ergebnisse.


\section{Erweiterung des Begriffs `Test Non-Stop'}
Nicht nur End to End Tests sondern Testen in Produktion\\
Erstens: Ansatz von Google mit Machine Learning Ansätze

\section{Priorisierung und Erweiterung von Unit-Tests}
Unit-Tests müssen noch mehr an Wichtigkeit gewinnen im agilen Umfeld \cite{linz_testing_2014}, auch Google Testing Blog zitieren. Ich schlage eine noch extremere Test-Pyramide vor (unproportional GUTE Unit-Tests, basierend auf MBT).\\
Sogar GUI Tests können als Unit-Test angesehen werden, mocken von Daten und Umgebungen --> Verweis zu Service-Virtualisierungen in großen Tools wie HP UFT\\
Klarstellen, dass End2End Systemtests extrem zeitraubende Unterfangen sind aber gleichzeitig sehr schnell gebrochen werden können (durch Technologieänderungen, Umstellungen der GUU/Plattform usw).

\section{Kein Modellformat hat sich etabliert}
Die Auswahl des Modellformats und die zugrunde liegende Notation ist eine fundamentale Entscheidung, bei der Einführung von modellbasiertem Testen. Meist geht die Auswahl eines MBT-Tools mit der Wahl des Modellformats einher. Die Ursprünge dieser Werkzeuge (\todo{Ref zu Tool Marktübersicht}), für welche Art von Softwareprojekten diese zumeist eingesetzt wurden, sind unschwer an der unterstützten Modellnotation zu erkennen.  In der Praxis besteht eine Vielzahl von Notationen und Formaten die auf sehr unterschiedlichen Prinzipien beruhen. Eine Transformation von einem Modellformat in ein anderes ist kaum möglich (beispielsweise lässt sich ein Modell in B Notation nur mit genauer Kenntniss des SUT in Graphwalker-Notation konvertieren.)\\
Erschwerend kommt hinzu, dass gerade die großen kommerziellen Tools (wie Leiros Smartesting \footnote{Leiros Smartesting Produktseite \url{http://www.smartesting.com/en/}} oder Conformiq's MBT Toolsuite \footnote{Conformiq Produktseite \url{https://www.conformiq.com/products/}}) vollständig auf nicht-portable Eigenentwicklungen setzen. Diese Hersteller versuchen gezielt große Software-Projekte in ihr Ökosystem einzusperren, verpassen es aber die Branche der MBT-Tools voranzubringen. Sie vermarkten ihre Modellformate als Produktgeheimnis und verhindern dadurch die Entwicklung eines Standards der modellbasiertes Testen viel stärker in Erscheinung treten ließe. Weiters lassen sich Fähigkeiten der Tools kaum vergleichen. Während die Firma Conformiq ohne weiteres Testversionen ihrer vollständigen Toolsuite anbietet, und die Open-Source Tools ohnehin frei evaluierbar sind, verlangt die Firma Leiros knapp 1000€ für eine akademische Lizenz. Auch auf Nachfrage des Autors wurde keine Demo- oder Testversion zum Zweck dieser Diplomarbeit zur Verfügung gestellt. Anhand offizieller Produktbeschreibungen und Quellen lässt sich nicht herausfinden wie sich der Workflow mit dem genannten Tool verhält, geschweige denn wie die zugrunde liegende Modellierung funktioniert.\\
Die Bemühungen der OMG mit dem UML Testing Profile eine einheitliche Sprache für modellbasiertes Testen zu schaffen ist ebenfalls nicht gelungen. Die Zahl der Publikationen die sich auf das UML Testing Profile stützen ist seit 2007 zwar konstant \footnote{Nach Recherche auf einschlägigen Portalen wie \url{http://ieeexplore.ieee.org/}} trotzdem bietet kein, dem Autor bekanntes, Tool Unterstützung dafür. Zwar schlägt die offizielle Publikation der OMG\cite{_model-driven_2007} manuelle Vorgehensweisen vor um UTP doch zu benutzen (siehe Kapitel \ref{sec:utp}), dies ist im Umfeld großer agiler Softwareprojekte aber nicht praktikabel. Zu viele Fehler könnten bei der manuellen Transformation von UTP-Diagramme passieren.


Kein Modellformat hat sich wirklich durchgsetzt, Werkzeuge verwenden oft proprietäre oder nicht-standardisierte Formate. Die Gefahr in einer Sackgasse zu landen ist groß. Wie schon Simon Mills\cite{graham_experiences_2012} bezüglich Skriptsprachen im Kontext von automatisierten GUI-Tests berichtet, ist es sehr vorteilhaft sich in keine Abhängigkeit mit einem Tool-Hersteller zu bringen, obwohl das Tool zur Zeit verwendet wird.
\fi\makeatother